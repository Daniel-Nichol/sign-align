% \pagebreak[4]
% \hspace*{1cm}
% \pagebreak[4]
% \hspace*{1cm}
% \pagebreak[4]

\chapter{Background and Model Selection}
\ifpdf
    \graphicspath{{Chapter1/Chapter1Figs/PNG/}{Chapter1/Chapter1Figs/PDF/}{Chapter1/Chapter1Figs/}}
\else
    \graphicspath{{Chapter1/Chapter1Figs/EPS/}{Chapter1/Chapter1Figs/}}
\fi

\section{Hidden Markov Models}
A hidden Markov Model (HMM) is (following the definitions in ~\cite{rabiner1989tutorial}) a doubly stochastic process which consists of an underlying discrete Markov chain with state set $S = \{S_1, \dots, S_n \}$ and stochastic matrix $A=[a_{i,j}]_{N\times N}$ where 
\begin{equation*}
a_{i,j} = \mathbb{P}(q_{t+1} = S_i | q_t = S_j) \text{  for each  } 1 \leq i,j \leq N
\end{equation*}
which is hidden from an observer in the respect that one cannot directly observe the current state of the Markov chain. Instead we have a collection of $M$ observable symbols, say $V = \{ v_1, \dots, v_M\}$, which may be observed with probability dependent on the state underlying Markov chain. 

We encode these so-called \emph{emission probabilities} in a matrix $B = [b_j(k)]_{N\times M}$ where
\begin{equation*}
b_j(k) = \mathbb{P}[v_k \text{ occurs at time }t | q_t = S_j] 
\end{equation*}
Now if we take an initial state distribution $\bm{\pi} = [\pi_1, \dots, \pi_N]$ for our Markov chain with
\begin{equation*}
\pi_i= \mathbb{P}[q_1 = S_i]
\end{equation*}
Then the HMM generates a sequence of observations
\begin{equation*}
\bm{O} = O_1, O_2, \ldots , O_T
\end{equation*}
by the following process:
\begin{enumerate}
\item Choose an initial state $q_1 = S_i$ stochastically from the initial state distribution $\bm{\pi}$
\item For $t=1$ to $T$
\begin{enumerate}
\item[i.] Choose $O_t = v_k$ according to the distribution $b_i(k)$ of the current state $S_i$
\item[ii.] Stochastically transition to a new state $S_j$ from $S_i$ according to $A$ 
\end{enumerate}
\end{enumerate}

Note now that a hidden Markov Model is entirely determined by the transition matrix $A$, the emissions matrix $B$ and the initial distribution $\bm{\pi}$ (noting that the dimensions $N$ and $M$ are encoded in the dimensions of the matrices) hence for convenience we may denote a HMM by
\begin{equation*}
\lambda = (A,B, \bm{\pi})
\end{equation*}

Hidden Markov Models were first introduced by in a series of papers by L.E Baum and others in the late 1960s ~\citep{baum1966statistical,baum1970maximization} and have been since used to study handwriting recognition~\citep{bunke1995off}, speech recognition ~\citep{juang1991hidden, jelinek1998statistical}, natural language modelling~\citep{manning1999foundations, jurafsky2002speech} and biological processes ~\citep{krogh1994hidden, durbin1998biological, lio1998models}.
\\
\\
Given these defintions there exist three basic problems which form the basis of using HMMs as a tool for machine learning
\begin{enumerate}
\item Given an observation sequence $\bm{O} = O_1,O_2,\dots,O_T$ and a HMM $\lambda$, what is $\mathbb{P}(\bm{O}|\lambda)$ - the probability that the observation sequence $\bm{O}$ was produced by $\lambda$?
\item Given a HMM $\lambda$ and an observation sequence $\bm{O} = O_1,O_2,\dots,O_T $ what is the state sequence of the underlying Markov chain in $\lambda$ which is most likely to have generated $\bm{O}$?
\item Given an observation sequence $\bm{O}$, how to do we choose the parameters of $\lambda = (A,B,\bm{\pi})$ which best optimise $\mathbb{P}(\bm{O}|\lambda)$?
\end{enumerate}

In fact the problem of sign language gesture recognition can be seen as instance of problems 3 and 1. First we take for each gesture $g$ a set of training data which are recordings of the joints in 3D space. This training data forms a collection of observation sequences $\bm{O_1}, \dots, \bm{O_n}$ which are used in a solution to problem 3 to parameterise a hidden Markov Model $\lambda_g$ to model the gesture.

Then given a fresh observation sequence $\bm{O}$ we can use a solution to problem 1 to compute $\mathbb{P}[\bm{O} | \lambda_g]$ for each $g$. We can then take the gesture $g$ for which this probability is maximised and, provided the probability exceeds some threshold, conclude that the gesture $g$ corresponds to the observation sequence $\bm{O}$. 

\subsection{The Forward and Backward Variables}
Suppose we are given an observation sequence $\bm{O} = O_1,O_2,\dots,O_T$ and a HMM $\lambda$ and we wish to solve the evaluation problem (problem 1). ~\citet{rabiner1989tutorial} notes that a direct computation of $\mathbb{P}[\bm{O}|\lambda]$ will take time order $\mathcal{O}(2TN^T)$ which is infeasible even for moderate values of $N$ and $T$. Instead we use a dynamic programming approach, the forward algorithm, introduced in~\citet{baum1968growth} and ~\cite{baum1970maximization}.

Define the \emph{forward variables} $\alpha_t(i)$ for each $1\leq t \leq T$ and $1 \leq i \leq N$ by
\begin{equation*}
\alpha_t(i) = \mathbb{P}[O_1, \dots, O_t, q_t=S_i | \lambda]
\end{equation*}
Then note these can be inductively computed by the following procedure 
\begin{align*}
&\alpha_1(i) = \pi_ib_i(O_1)\\
&\alpha_{t+1}(j) = \left[ \sum_{i=1}^{N} \alpha_{t}(i)a_{ij} \right]b_j(O_{t+1}) &\text{ for $1\leq t \leq T-1$ and $1 \leq j \leq N$}
\end{align*}
Finally we have
\begin{align*}
\mathbb{P}[\bm{O} | \lambda] &= \sum_{i=1}^N \mathbb{P}[\bm{O}, q_T = S_i | \lambda] = \sum_{i=1}^N \alpha_T(i)
\end{align*}
The set of forward variables can be computed by dynamic programming in $\mathcal{O}(N^2T)$ time and hence we can compute $\mathbb{P}[\bm{O} | \lambda]$ in this time -  a significant improvement over direct computation. This algorithm is the \emph{forward algorithm} for evaluation in HMMs. 

Symmetrically to the forward variables we can define a collection of \emph{backward variables} by
\begin{equation*}
\beta_t(i) = \mathbf{P}[O_{t+1}, O_{t+2}, \dots, O_T | q_t = S_i, \lambda]
\end{equation*}
Which gives the probability of a partial observation sequence from a time $t$ given the state of the HMM $\lambda$ at time t. These too can be computed iteratively as
\begin{align*}
&\beta_T(i) = 1 &\text{ for each } 1 \leq i \leq N \\
&\beta_t(i)  = \sum_{j=1}^N a_{i,j}b_j(O_{t+1})\beta_{t+1}(j) &\text{ for } t = T-1, T-2,\dots,1 \text{ and } 1 \leq i \leq N \\
\end{align*}
and these too can be computed by dynamic programming in $\mathcal{O}(N^2T)$ time. The backward variables are not needed in the solution to the evalutation problem but are used in the following section to re-parameterise hidden Markov Models. 

\subsection{Forward-Backward Algorithm}
The problem of parameterising a hidden Markov Model $\lambda$ is considerably more difficult than the other two problems of HMMs. In fact it is known that there is no analytic solution which provides a model $\lambda$ to maximise the probability of some observation sequence $\bm{O}$. Instead we must use local optimisation methods which given an initial parameterisation $\lambda_0$ iteratively improve it until $\mathbb{P}[\bm{O} | \lambda]$ reaches a local maximum. 

We will use a modified version of the \emph{Baum-Welch algorithm} introduced by~\citet{baum1970maximization} called the \emph{forward-backward algorithm}~\citep{rabiner1989tutorial} which is reproduced below. This algorithm is an instance of an expectation-maximization algorithm~\cite{moon1996expectation} which is a general technique used to determine maximum likelihood estimators in a number of machine learning models~\cite{bishop2006pattern}. \\
\\
For $t \in \{1, \dots, T-1\}$ and $i,j \in \{1, \dots, N\}$ define
\begin{equation*}
\gamma_t(i,j) = \mathbb{P}[q_t = S_i, q_{t+1}=S_j| \bm{O}, \lambda]
\end{equation*}
such that $\gamma_t(i,j)$ is the probability of being in state $S_i$ at time $t$ and transitioning to $S_j$ at the next time step given the HMM $\lambda$ and the observation sequence $\bm{O}$. The by definiton of the foward and backward variables we have
\begin{align*}
\gamma_t(i,j) &= \frac{\alpha_t(i)a_{ij}b_j(O_{t+1})\beta(j)}{\mathbb{P}[\bm{O}|\lambda]}	
\end{align*}
We can define for each $1 \leq t \leq T-1$ and each $1 \leq i \leq N$ the probability of being in state $i$ at time $t$ by
\begin{equation*}
\gamma_t(i) = \mathbb{P}[q_t = S_n | \bm{O}, \lambda] = \frac{\alpha_t(i)\beta_t(i)}{\mathbb{P}[\bm{O}|\lambda]}
\end{equation*}
and then we have
\begin{equation*}
\gamma_t(i) = \sum_{j=1}^N \gamma_t(i,j)
\end{equation*}
Now using these equations we can compute
\begin{align*}
&\sum_{t=1}^{T-1} \gamma_t(i) = \text{The expected number of transitions from $S_i$} \\
&\sum_{t=1}^{T-1} \gamma_t(i,j) =  \text{The expected number of transitions from $S_i$ to $S_j$}\\
\end{align*}
which can be used to reparameterise the model $\lambda = (A,B,\bm{\pi})$ as follows, set
\begin{align*}
\overline{\pi} &= \text{the expected number of times in state $S_i$ at time $1$} = \gamma_1(i) \\
\overline{a_{ij}} &= \frac{\text{the expected number if transitions from $S_i$ to $S_j$}}{\text{expected number of transitions from state $S_i$}} \\ 
				&= \frac{\sum_{t=1}^{T-1} \gamma_t(i,j)}{\sum_{t=1}^{T-1} \gamma_t(i)} \\
\overline{b}_j(k) &= \frac{\text{the expected number of times in state $S_j$ observing symbol $v_k$}}{\text{the expected number of times in state $S_j$}} \\
				&=\frac{\sum_{\stackrel{t=1}{O_t=v_k}}^{T}\gamma_t(j)}{\sum_{t=1}^T \gamma_t(j)}
\end{align*}
Then if denote $\overline{\lambda} = (\overline{A},\overline{B},\overline{\bm{\pi}})$ then it has been proven ~\citep{levinson1983introduction, baum1968growth} that either
\begin{enumerate}
\item $\mathbb{P}[\bm{O}|\overline{\lambda}] > \mathbb{P}[\bm{O}|\lambda]$ or
\item $\lambda$ is locally maximized with respect to $\mathbb{P}[\bm{O}|\lambda]$ and $\overline{\lambda} = \lambda$
\end{enumerate}
It follows that given an initial HMM $\lambda$ we may improve it to a locally optimal model for some observation sequence $\bm{O}$ via the following local search:
\begin{enumerate}
\item Whilst $\mathbb{P}[\bm{O}|\lambda]$ increases:
	\begin{enumerate}
		\item[i.] Compute the $\alpha_t{i}$, $\beta_t(I)$, $\gamma_t(i,j)$ and $\gamma_t(i)$
		\item[ii.] Compute the new parameters $\overline{A} = [\overline{a}_{ij}]$, $\overline{B} = [\overline{b}_j(k)]$ and $\overline{\bm{\pi}} = [\overline{\pi_1}, \dots, \overline{\pi_N}]$
		\item[iii.] Set $\lambda := \overline{\lambda}$
	\end{enumerate}
\item return $\lambda$
\end{enumerate}
Note that this algorithm may not actually terminate. In practice we threshold the increase of $\mathbb{P}[\bm{O}|\lambda]$ to ensure termination in a reasonable time. Further, in practice we will not have single observation sequence but rather a collection, perhaps from a variety of signers of different heights, genders, ages or dialects. In the implementation of our HMM framework we will modify this algorithm to work for multiple observation sequences. This will again increase the time complexity of the algorithm, however this problem is not so significant as in practice we will use this algorithm once per sign and save the parameterisations.

\subsection{Limitations of HMMs}
A significant limitation of hidden Markov Models is that they can have only a finite set $V$ of possible observations. In practice this can prevent us using a HMM to model a specific gesture exactly. Take for example the problem of detecting the gesture of a circle drawn on a 2D plane. We might create a hidden Markov Model in which the states of the Markov chain represent some specific points on the plane and want our matrix $B$ to be such that
\begin{align*}
&b_j(\bm{p}) = \mathbb{P}[\text{the pen is at point $\bm{p}$ at time $t$ } | q_t = A_j] &\text{ for each point $\bm{p}\in\mathbb{R}^2$}
\end{align*}
However as the plane is continuous and $V$ is finite this is not possible. One solution is the discretise the plane and have only a finite (but large) set of observation symbols. This method has been used with some success to distuingish between gestures with large variations, for example different tennis strokes~\citep{yamato1992recognizing}. However without sufficiently fine grained discretisation, which will severely impact the efficiency of our algorithms, our observation sequences will suffer from signal degradation. As we plan to model signed languages, in which certain subtle changes to a gesture can change the meaning {\color{green}[GIVEN AN EXAMPLE]}, it is likely signal degradation will impact on the accuracy of our system.

\section{Continuous Distribution HMMs}

% ------------------------------------------------------------------------


%%% Local Variables: 
%%% mode: latex
%%% TeX-master: "../thesis"
%%% End: 
